%%%%%%%%%%%%%%%%%%%%%%%%%%%%%%%%%%%%%%%%%
% Journal Article
% LaTeX Template
% Version 1.4 (15/5/16)
%
% This template has been downloaded from:
% http://www.LaTeXTemplates.com
%
% Original author:
% Frits Wenneker (http://www.howtotex.com) with extensive modifications by
% Vel (vel@LaTeXTemplates.com)
%
% License:
% CC BY-NC-SA 3.0 (http://creativecommons.org/licenses/by-nc-sa/3.0/)
%
%%%%%%%%%%%%%%%%%%%%%%%%%%%%%%%%%%%%%%%%%

%----------------------------------------------------------------------------------------
%	PACKAGES AND OTHER DOCUMENT CONFIGURATIONS
%----------------------------------------------------------------------------------------

\documentclass[twoside,twocolumn]{article}


\usepackage{blindtext} % Package to generate dummy text throughout this template 

\usepackage[utf8]{inputenc}

\usepackage[sc]{mathpazo} % Use the Palatino font
\usepackage[T1]{fontenc} % Use 8-bit encoding that has 256 glyphs
\linespread{1.05} % Line spacing - Palatino needs more space between lines
\usepackage{microtype} % Slightly tweak font spacing for aesthetics

\usepackage[italian]{babel} % Language hyphenation and typographical rules

\usepackage[hmarginratio=1:1,top=32mm,columnsep=20pt]{geometry} % Document margins
\usepackage[hang, small,labelfont=bf,up,textfont=it,up]{caption} % Custom captions under/above floats in tables or figures
\usepackage{booktabs} % Horizontal rules in tables

\usepackage{lettrine} % The lettrine is the first enlarged letter at the beginning of the text

\usepackage{enumitem} % Customized lists
\setlist[itemize]{noitemsep} % Make itemize lists more compact

\usepackage{abstract} % Allows abstract customization
\renewcommand{\abstractnamefont}{\normalfont\bfseries} % Set the "Abstract" text to bold
\renewcommand{\abstracttextfont}{\normalfont\small\itshape} % Set the abstract itself to small italic text

\usepackage{titlesec} % Allows customization of titles
\renewcommand\thesection{\Roman{section}} % Roman numerals for the sections
\renewcommand\thesubsection{\roman{subsection}} % roman numerals for subsections
\titleformat{\section}[block]{\large\scshape\centering}{\thesection.}{1em}{} % Change the look of the section titles
\titleformat{\subsection}[block]{\large}{\thesubsection.}{1em}{} % Change the look of the section titles

\usepackage{fancyhdr} % Headers and footers
\pagestyle{fancy} % All pages have headers and footers
\fancyhead{} % Blank out the default header
\fancyfoot{} % Blank out the default footer
\fancyhead[C]{Metodi Computazionali della Fisica} % Custom header text
\fancyfoot[RO,LE]{\thepage} % Custom footer text

\usepackage{titling} % Customizing the title section

\usepackage{hyperref} % For hyperlinks in the PDF

%----------------------------------------------------------------------------------------
%	TITLE SECTION
%----------------------------------------------------------------------------------------

\setlength{\droptitle}{-4\baselineskip} % Move the title up

\pretitle{\begin{center}\Huge\bfseries} % Article title formatting
\posttitle{\end{center}} % Article title closing formatting
\title{Studio di algoritmi stocastici per la costruzione di quadrati magici} % Article title
\author{%
\textsc{Gabriele Bozzola} \\[1ex] % Your name
\normalsize Università degli Studi di Milano \\ % Your institution
\normalsize \href{mailto:bozzola.gabriele@gmail.com}{bozzola.gabriele@gmail.com} % Your email address
%\and % Uncomment if 2 authors are required, duplicate these 4 lines if more
%\textsc{Jane Smith}\thanks{Corresponding author} \\[1ex] % Second author's name
%\normalsize University of Utah \\ % Second author's institution
%\normalsize \href{mailto:jane@smith.com}{jane@smith.com} % Second author's email address
}
\date{Luglio 2016} % Leave empty to omit a date
\renewcommand{\maketitlehookd}{%
\begin{abstract}
\noindent In questo lavoro vengono presentati due algoritmi stocastici per costruire quadrati magici normali. Il primo è un algoritmo genetico realizzato utilizzando diverse funzioni di fitness e metodi di selezione, mentre il secondo è un algoritmo evolutivo, basato sul lavoro di Xie e Kang. 
\end{abstract}
}

%----------------------------------------------------------------------------------------

\begin{document}

% Print the title
\maketitle

%----------------------------------------------------------------------------------------
%	ARTICLE CONTENTS
%----------------------------------------------------------------------------------------

\section{Introduzione}

\lettrine[nindent=0em,lines=3]{U}n quadrato magico di ordine $ N $ è una matrice quadrata $ N\times N $ contenente tutti i numeri naturali distinti tale che la somma dei valori su ciascuna riga, colonna e diagonale sia sempre la stessa, detta \emph{numero magico}. Qualora i numeri che compaiono sono i primi $ N^2 $ allora si il quadrato è detto \emph{normale}. Gli algoritmi sono quindi implementati in Mathematica 8.

\subsection{Definizione e utilizzi}


\subsection{Proprietà}

\blindtext % Dummy text

\subsection{Stato dell'arte}

\blindtext % Dummy text

\section{Algoritmi genetici}

\subsection{Funzioni di fitness}

\subsection{Metodi di selezione}

\subsection{Implementazione in Mathematica}

\subsection{Conclusioni}

\section{Algoritmo evolutivo}
Per superare i problemi legati all'applicazione di un algoritmo genetico per la costruzione di quadrati magici normali si adottano due accorgimenti:
\begin{enumerate}
	\item Si elimina la fase di crossover e si aumenta il numero di mutazioni effettuate sul singolo individuo.
	\item Si effettuano controlli sistematici quando l'algoritmo comincia ad essere in condizioni di stallo. 
\end{enumerate}
Siccome ora non vi sono più crossover non è necessario lavorare con una popolazione composta da numero elevato di individui, ma se ne utilizza uno solo, il quale produce un numero fissato di figli. 

In questo lavoro è stata implementato un'algoritmo basato su quello proposto da  Xie e Kang in \cite{XieKang:2003}.

\subsection{Algoritmo di Xie-Kang}

Tale algoritmo implementa i due miglioramenti esposti all'inizio di questa sezione e aggiunge un ulteriore contributo fondamentale: \emph{la congettura della costruzione a due fasi}.

\subsubsection{Congettura della costruzione a due fasi}
Una matrice composta da numeri naturali differenti $ N\times N $ è detta \emph{quadrato semimagico} se è un quadrato magico a meno delle diagonali, ovvero se la somma ddei valori su tutte le righe e su tutte le colonne è uguale al numero magico. Un quadrato semimagico è normale se le sue entrate sono tutti i numeri da $ 1 $ a $ N^2 $. 

La congettura della costruzione a due fasi di Xie e Kang afferma che un quadrato semimagico è sempre completabile ad un quadrato magico utilizzando un numero finito di permutazioni di righe e di colonne oppure di rettificazioni locali. 

\subsection{Metodi di selezione}

\subsection{Mutazioni}

\subsubsection{Mutazioni puntuali}

\subsection{Rettificazioni locali}

\subsection{Implementazione in Mathematica}

%------------------------------------------------

\section{Methods}

Maecenas sed ultricies felis. Sed imperdiet dictum arcu a egestas. 
\begin{itemize}
\item Donec dolor arcu, rutrum id molestie in, viverra sed diam
\item Curabitur feugiat
\item turpis sed auctor facilisis
\item arcu eros accumsan lorem, at posuere mi diam sit amet tortor
\item Fusce fermentum, mi sit amet euismod rutrum
\item sem lorem molestie diam, iaculis aliquet sapien tortor non nisi
\item Pellentesque bibendum pretium aliquet
\end{itemize}
\blindtext % Dummy text

Text requiring further explanation\footnote{Example footnote}.

%------------------------------------------------

\section{Risultati}

\begin{table}
\caption{Example table}
\centering
\begin{tabular}{llr}
\toprule
\multicolumn{2}{c}{Name} \\
\cmidrule(r){1-2}
First name & Last Name & Grade \\
\midrule
John & Doe & $7.5$ \\
Richard & Miles & $2$ \\
\bottomrule
\end{tabular}
\end{table}

\blindtext % Dummy text

\begin{equation}
\label{eq:emc}
e = mc^2
\end{equation}

\blindtext % Dummy text

%------------------------------------------------

\section{Conclusioni}

\subsection{Subsection One}

A statement requiring citation \cite{XieKang:2003}.
\blindtext % Dummy text

\subsection{Subsection Two}

\blindtext % Dummy text


\section{Appendice}

Esempio rettificazione su riga 1:

Prima: 

\begin{tabular}{|c|c|c|}
	\hline 
	\rule[-1ex]{0pt}{2.5ex} 1 & 5 & 6 \\ 
	\hline 
	\rule[-1ex]{0pt}{2.5ex} 4 & 3 & 8 \\ 
	\hline 
	\rule[-1ex]{0pt}{2.5ex} 2 & 7 & 9 \\ 
	\hline 
\end{tabular} 

Dopo: 

\begin{tabular}{|c|c|c|}
	\hline 
	\rule[-1ex]{0pt}{2.5ex} 1 & 5 & 9 \\ 
	\hline 
	\rule[-1ex]{0pt}{2.5ex} 4 & 3 & 8 \\ 
	\hline 
	\rule[-1ex]{0pt}{2.5ex} 2 & 7 & 6 \\ 
	\hline 
\end{tabular} 


Esempio di rettificazione diagonale 2:

Prima:

\begin{tabular}{|c|c|c|}
	\hline 
	\rule[-1ex]{0pt}{2.5ex} 9 & 5 & 1 \\ 
	\hline 
	\rule[-1ex]{0pt}{2.5ex} 3 & 4 & 2 \\ 
	\hline 
	\rule[-1ex]{0pt}{2.5ex} 8 & 7 & 6 \\ 
	\hline 
\end{tabular} 

Dopo:

\begin{tabular}{|c|c|c|}
	\hline 
	\rule[-1ex]{0pt}{2.5ex} 9 & \textbf{4} & \textbf{2} \\ 
	\hline 
	\rule[-1ex]{0pt}{2.5ex} 3 & \textbf{5} & \textbf{1} \\ 
	\hline 
	\rule[-1ex]{0pt}{2.5ex} 8 & 7 & 6 \\ 
	\hline 
\end{tabular} 


%----------------------------------------------------------------------------------------
%	REFERENCE LIST
%----------------------------------------------------------------------------------------

\begin{thebibliography}{99} % Bibliography - this is intentionally simple in this template

\bibitem{XieKang:2003}
Xie, T. e Kang, L. (2003).
\newblock An Evolutionary Algorithm for Magic Squares.
\newblock {\em The 2003 Congress on Evolutionary Computation, 2003}.

 
\end{thebibliography}

%----------------------------------------------------------------------------------------

\end{document}
