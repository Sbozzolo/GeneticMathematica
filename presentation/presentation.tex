\documentclass[10pt]{beamer}

\usetheme[titleformat = smallcaps, block = fill]{metropolis}
\usepackage{appendixnumberbeamer}

\usepackage[italian]{babel}

\usepackage{booktabs}
\usepackage[scale=2]{ccicons}

%\usepackage{pgf-pie}

%\usepackage{pgfplots}
%\usetikzlibrary{external}
%\tikzexternalize
%\tikzsetexternalprefix{images/}

\usepackage{amsmath}

%Animazioni
\usepackage{animate}
\usepackage{graphicx}
\usepackage{xmpmulti}

%Togliere tablella
\usepackage{caption}

%Overlay colorato
\usepackage{colortbl}

%Codice di Mathematica
\usepackage{listings}

%Per colorare i commenti di LateX
\lstset{escapeinside={<@}{@>}}


%Grafici
\usepackage{subfig}
\usepackage{siunitx}
\usepackage{pgfplots}
\usepgfplotslibrary{units}
\usepgfplotslibrary{dateplot}

%Flowchart
\usepackage{makeshape}
\usepackage{flowchart}
\usetikzlibrary{arrows}

\usepackage{xspace}
\newcommand{\themename}{\textbf{\textsc{metropolis}}\xspace}

%Per diminuire l'indentazione di itemize
\newlength\origleftmargini
\setlength\origleftmargini\leftmargini
\setbeamertemplate{itemize/enumerate body begin}{\setlength{\leftmargini}{7.5pt}}

\let\oldblock\block
\let\oldendblock\endblock
\def\block{\begingroup \setbeamertemplate{itemize/enumerate body begin}{\setlength{\leftmargini}{10pt}} \oldblock}
\def\endblock{\oldendblock \endgroup}

\let\oldalertblock\alertblock
\let\oldendalertblock\endalertblock
\def\alertblock{\begingroup \setbeamertemplate{itemize/enumerate body begin}{\setlength{\leftmargini}{10pt}} \oldalertblock}
\def\endalertblock{\oldendalertblock \endgroup}

\renewcommand*{\thefootnote}{\fnsymbol{footnote}}


\title{Studio di algoritmi stocastici per la costruzione di quadrati magici}
\subtitle{Metodi Computazionali della Fisica}
\date{13 Luglio 2016}
\author{Gabriele Bozzola \\ Matricola 882709}
\institute{Università degli studi di Milano}
% \titlegraphic{\hfill\includegraphics[height=1.5cm]{log	o.pdf}}

\begin{document}

	\maketitle

%\section{il problema}

\begin{frame}[fragile]{quadrati magici}
  \begin{columns}[T,onlytextwidth]
    \column{0.5\textwidth}

    \begin{block}{Definizione}
    	Un \alert{quadrato magico normale} $ N\times N $ è una matrice di ordine $ N $ contenente tutti i numeri naturali da 1 a $ N^2 $ tali che la somma di tutti gli elementi sulle righe, sulle colonne e sulle diagonali sia sempre la stessa, detta numero magico.
    \end{block}

	\column{0.5\textwidth}
    \begin{figure}
    	\centering
%    	\tikzsetnextfilename{quadrato_magico}
    	\begin{tikzpicture}[scale = 1]
    	\node at (-1,0)  {};

    	\draw (0,0) grid (3,3);

    	\node at (1.5, 1.5) {\large 5};
    	\node at (0.5, 0.5) {\large 2};
    	\node at (1.5, 0.5) {\large 9};
    	\node at (0.5, 2.5) {\large 6};
    	\node at (0.5, 1.5) {\large 7};
    	\node at (2.5, 2.5) {\large 8};
    	\node at (1.5, 2.5) {\large 1};
    	\node at (2.5, 0.5) {\large 4};
    	\node at (2.5, 1.5) {\large 3};

    	\draw[->, thick] (3.25, 2.5) -- (3.5, 2.5) node[right] {\large 15};
    	\draw[->, thick] (3.25, 1.5) -- (3.5, 1.5) node[right] {\large 15};
    	\draw[->, thick] (3.25, 0.5) -- (3.5, 0.5) node[right] {\large 15};
    	\draw[->, thick] (0.5, -0.25) -- (0.5, -0.5) node[below] {\large 15};
    	\draw[->, thick] (1.5, -0.25) -- (1.5, -0.5) node[below] {\large 15};
    	\draw[->, thick] (2.5, -0.25) -- (2.5, -0.5) node[below] {\large 15};
    	\draw[->, thick] (3.25, -0.25) --(3.5, -0.5);
    	\node at (3.75, -0.75) {\large 15};
    	\draw[->, thick] (-0.25, -0.25) --(-0.5, -0.5);
    	\node at (-0.75, -0.75) {\large 15};
    	\end{tikzpicture}
    \end{figure}
 \end{columns}
\end{frame}

\begin{frame}{quadrati magici -- proprietà}
	\begin{alertblock}{Teorema di esistenza}
		$ \forall N \in \mathbb{N} - \{2\} $ è sempre possibile costruire almeno un quadrato magico normale.
	\end{alertblock}

	\begin{alertblock}{Formula per la costante magica}
		$ \forall N \in \mathbb{N} - \{2\} $ la costante magica $ m.v. $ di un quadrato di ordine $ N $ è:
		\begin{equation*}\label{equa:costante_magica}
		 \mathit{m.v.} = \frac{1}{2} N (N^2 + 1)
		\end{equation*}
	\end{alertblock}
\end{frame}


\setbeamertemplate{numero}{Dati ottenuti con metodi statistici da Trump W.}
\begin{frame}{quadrati magici -- numero}
	\begin{columns}

		\column{0.33\textwidth}
	\begin{block}{Numero di quadrati magici di ordine $ N $}
		Il numero di quadrati magici è noto con precisione solo per $ N<6 $. \\
		La percentuale sul totale dei possibili quadrati tende a zero per $ N $ che tende a $ +\infty $.
	\end{block}

		\column{0.66\textwidth}
		\begin{tabular}{cccc} \toprule
			$ N $ &       $ N_{ms} $        &       $ N_{ns} $        &  $ \si{\percent} $   \\ \midrule
			2   &            0            &    $ \sim  10^{1} $     &          0           \\
			3   &            1            &     $ \sim 10^{5} $     &   $ \sim 10^{-5} $   \\
			4   &           880           &    $ \sim 10^{12} $     &   $ \sim 10^{-7} $   \\
			5   &       275 305 224       &    $ \sim 10^{24} $     &  $ \sim 10^{-18} $   \\
			6   &    $ \sim 10^{19} $     &    $ \sim 10^{41} $     &  $ \sim 10^{-22} $   \\
			20   &    $ \sim 10^{744} $    &    $ \sim 10^{868} $    &  $ \sim 10^{-124} $  \\
			35   &   $ \sim 10^{2992} $    &   $ \sim 10^{3252} $    &  $ \sim 10^{-250} $  \\
			50   &   $ \sim 10^{7000} $    &   $ \sim 10^{7410} $    &  $ \sim 10^{-410} $\\ \bottomrule
			%	10000 & $ \sim 10^{756373366} $ & $ \sim 10^{756570555} $ & $ \sim 10^{-197189} $ \\
		\end{tabular}
	\end{columns}
	\centering
	Trovare quadrati magici è \alert{difficile}.\footnote{Dati ottenuti con metodi statistici da Trump W. e con errore inferiore a $ \SI{1}{\percent} $.}
\end{frame}

\begin{frame}{quadrati magici - metodi di costruzione deterministici}

	\begin{columns}[T,onlytextwidth]

	\column{0.5\textwidth}

	\begin{itemize}
		\item Per quadrati di ordine dispari: metodo de la Loumbre, metodo di Conway, metodo Pheru, \dots
		\item Per quadrati pari: metodo Medjing, \dots
		\item Per quadrati singolarmente pari ($ N $ è multiplo di quattro): metodo LUX, metodo Strachey, \dots
	\end{itemize}

	\column{0.5\textwidth}

%	    	\transduration<0-95>{0}
%	    	\multiinclude[<+->][format=png, graphics={width=\textwidth}]{siamesegif/SiameseMethod}

	\animategraphics[loop,controls,width=\columnwidth]{8}{siamesegif/SiameseMethod-}{0}{95}

	\end{columns}

\end{frame}

\begin{frame}{quadrati magici - problemi di questi metodi}
  \begin{itemize}
  	\item Costruiscono sempre il medesimo quadrato
  	\item Non sono generalizzabili ad altri tipi di quadratici magici
  \end{itemize}
  \centering
  I metodi stocastici risolvono questi problemi.
\end{frame}

%\section{algoritmi genetici}

\begin{frame}{algoritmo genetico - funzionamento}

	\begin{columns}[T,onlytextwidth]

	\column{0.5\textwidth}
    \begin{block}{}
    	\parbox{0.98\columnwidth}{
    		Gli algoritmi genetici implementano il principio darwiniano di \alert{sopravvivenza del più adatto}.

    		Gli individui:
    		\begin{itemize}
    			\item sono possibili soluzioni del problema
    			\item sono classificati in base alla loro \alert{fitness}, che quantifica quanto si avvicinano alla soluzione reale
    			\item si riproducono in maniera sessuata (\alert{crossover})
    			\item possono subire \alert{mutazioni}
    		\end{itemize}

    	}
    \end{block}

    \column{0.5\textwidth}
	\begin{figure}[htbp]
		\centering
		\resizebox*{!}{1.25\columnwidth}{
		\begin{tikzpicture}
		\def\smbwd{0.7\columnwidth}
		\node (start) at (0,0) [draw, terminal,
		minimum width=\smbwd] {INIZIO};
		\node (generapop) at (0, -1) [draw, process,
		minimum width=\smbwd] {GENERA POPOLAZIONE};
		\node (selgen) at (0, -2) [draw, process,
		minimum width=\smbwd] {SELEZIONA GENITORI};
		\node (crossover) at (0, -3) [draw, process,
		minimum width=\smbwd] {CROSSOVER};
		\node (muta) at (0, -4) [draw, process,
		minimum width=\smbwd] {MUTA};
		\node (trovato) at (0,-6) [draw, decision,
		] {TROVATO?};
		\node (end) at (0,-8) [draw, terminal,
		minimum width=\smbwd] {FINE};

		\draw[->] (start) -- (generapop);
		\draw[->] (generapop) -- (selgen);
		\draw[->] (selgen) -- (crossover);
		\draw[->] (crossover) -- (muta);
		\draw[->] (muta) -- (trovato);
		\draw[->] (trovato) --node[left]{SI} (end);
		\draw[->] (trovato) --node[above]{NO} (-2.75, -6)
		-- (-2.75, -1.5) -- (0, -1.5);
		\end{tikzpicture}
	}
	\end{figure}

\end{columns}
\end{frame}

\begin{frame}{algoritmo genetico -- perché è adatto?}
	Perché il problema della costruzione dei quadrati magici è un buon problema da affrontare con gli algoritmi genetici?
	\begin{itemize}
		\item Lo spazio delle soluzioni è estremamente vasto
		\item I quadrati possono essere codificati in modo diretto come individui
		\item Il problema può essere formulato come ottimizzazione di una funzione di fitness
	\end{itemize}
\end{frame}

\begin{frame}{algoritmo genetico -- implementazione}
	Per implementare un algoritmo genetico bisogna a pensare a:
	\begin{itemize}
		\item Che funzione di fitness utilizzare?
		\item Come far selezionare i genitori?
		\item Come far riprodurre i quadrati?
		\item Come mutarli?
	\end{itemize}
\end{frame}

\begin{frame}{algoritmo genetico -- fitness}
	\begin{block}{Funzioni di fitness implementate}
	\parbox{0.98\columnwidth}{
		\begin{itemize}
		\item \texttt{totalSquared}: somma dei quadrati delle discrepanze delle somme di ogni linea dal valore magico
		\item \texttt{totalAbs}: somma dei moduli delle discrepanze delle somme di ogni linea dal valore magico
		\item \texttt{correctLines}: numero di linee magiche
		\end{itemize}
	}
	\end{block}
\end{frame}

\begin{frame}{algoritmo genetico -- metodi di selezione}
	\begin{block}{Metodi di selezione implementati}
	\parbox{0.98\columnwidth}{
	\begin{itemize}
		\item \texttt{fitnessProportionate}: probabilità di selezione proporzionale alla fitness
		\item \texttt{similarSquare}: probabilità di selezione dipendente dalla fitness e dalla distanza dal quadrato migliore
		\item \texttt{fittests}: alcuni individui non si riproducono, gli altri hanno uguale probabilità di selezione
		\item \texttt{elitism}: alcuni individui passano direttamente alla generazione successiva, i restanti vengono selezionati secondo uno dei precedenti metodi
	\end{itemize}
	}
	\end{block}
\end{frame}

\begin{frame}{algoritmo genetico -- metodi di riproduzione}

	\begin{columns}[T,onlytextwidth]

	\column{0.47\textwidth}

	\begin{block}{Metodi di crossover}
		\parbox{0.96\columnwidth}{
		Crossover a uno o due punti verticale o orizzontale. \\
		Se il crossover produce numeri doppi questi vengono sistemati \alert{casualmente}.
		}
	\end{block}

	\begin{block}{Metodi di mutazione}
		\parbox{0.96\columnwidth}{
			\begin{itemize}
				\item Scambio di una coppia
				\item Scambio di due colonne
				\item Scambio di due righe
				\item Permutazione di una riga
				\item Permutazione di una colonna
			\end{itemize}
		}
	\end{block}
	\column{0.5\textwidth}
	\vspace*{-15pt}
	\begin{figure}[!htbp]
		\centering
		\subfloat[Crossover orizzontale]{
			\begin{tikzpicture}[scale = 0.75]
			\draw (0,0) grid (3,3);
			\draw
			(0,1)[ultra thick] to (2,1)
			(2,1)[ultra thick] to (2,2)
			(2,2)[ultra thick] to (3,2);

			\draw[fill = gray, opacity = 0.30] (0,2) rectangle (3,3);
			\draw[fill = gray, opacity = 0.30] (0,1) rectangle (2,2);

			\node at (1.5, 1.5) {X};
                      \end{tikzpicture}
                    }\quad
                    \subfloat[Crossover verticale]{
                      \begin{tikzpicture}[scale = 0.75]
			\draw (0,0) grid (3,3);
			\draw
			(1,0)[ultra thick] to (1,1)
			(1,1)[ultra thick] to (2,1)
			(2,1)[ultra thick] to (2,3);

			\draw[fill = gray, opacity = 0.30] (0,0) rectangle (1,3);
			\draw[fill = gray, opacity = 0.30] (1,1) rectangle (2,3);

			\node at (1.5, 1.5) {X};
                      \end{tikzpicture}
                    }
                  \end{figure}
                  \begin{figure}[!htbp]
                    \centering
                    \subfloat[Crossover a due punti orizzontale]{
                      \begin{tikzpicture}[scale = 0.75]
			\draw (0,0) grid (3,3);
			\draw
			(0,1)[ultra thick] to (3,1)
			(0,2)[ultra thick] to (3,2);

			\draw[fill = gray, opacity = 0.30] (0,1) rectangle (3,2);

			\node at (0.5, 1.5) {X};
			\node at (2.5, 1.5) {X};
            \end{tikzpicture}
            }\quad
            \subfloat[Crossover a due punti verticale]{
            \begin{tikzpicture}[scale = 0.75]
               \draw (0,0) grid (3,3);
            \draw
                        (0,2)[ultra thick] to (1,2)
                        (1,2)[ultra thick] to (1,3)
                        (2,1)[ultra thick] to (2,0)
                        (2,1)[ultra thick] to (3,1);

            \draw[fill = gray, opacity = 0.30] (1,0) rectangle (2,3);
            \draw[fill = gray, opacity = 0.30] (0,0) rectangle (1,2);
            \draw[fill = gray, opacity = 0.30] (3,3) rectangle (2,1);

			\node at (0.5, 1.5) {X};
			\node at (2.5, 1.5) {X};
            \end{tikzpicture}
		   }
           \end{figure}
	\end{columns}
\end{frame}

\begin{frame}{algoritmo genetico -- risultati}
%		\parbox{0.98\columnwidth}{
			\alert{Nessuna} combinazione di fitness e metodi di selezione e crossover è riuscita a costruire quadrati di dimensioni superiori a $ 3\times3 $!
			\\ \hfill \\
			Motivo: lo spazio delle soluzioni non è \alert{connesso} rispetto a nessuna funzione di fitness.
%		}
\end{frame}

\begin{frame}{algoritmo genetico -- possibili miglioramenti}
	%		\parbox{0.98\columnwidth}{
	\begin{itemize}
		\item I crossover sono dannosi per giungere ad una soluzione. Non si possono eliminare?
		\item Non si può \emph{aiutare} l'algoritmo a convergere sbloccandolo nei momenti di stallo?
	\end{itemize}
	%		}
\end{frame}

%\section{algoritmo evolutivo}

\begin{frame}{algoritmo evolutivo - funzionamento}

	\begin{columns}[onlytextwidth]

	\column{0.5\textwidth}
	\begin{block}{}
		\parbox{0.95\columnwidth}{
			Gli algoritmi evolutivi sono particolari algoritmi genetici in cui:
			\vspace*{-15pt}
			\begin{itemize}
				\item \alert{Non} ci sono crossover
				\item Le mutazioni sono molto più sofisticate
				\item Sostanzialmente si lavora con \alert{un solo} individuo
			\end{itemize}
		}
	\end{block}

		\column{0.5\textwidth}
		\begin{figure}[htbp]
			\centering
			\resizebox*{!}{1.25\columnwidth}{
				\begin{tikzpicture}
				\def\smbwd{0.8\columnwidth}
				\node (start) at (0,0) [draw, terminal,
				minimum width=\smbwd] {INIZIO};
				\node (generapop) at (0, -1) [draw, process,
				minimum width=\smbwd] {GENERA PADRE};
				\node (selgen) at (0, -2) [draw, process,
				minimum width=\smbwd] {CLONA INDIVIDUO};
				\node (crossover) at (0, -3) [draw, process,
				minimum width=\smbwd] {MUTA POPOLAZIONE};						\node (muta) at (0, -4) [draw, process,
				minimum width=\smbwd] {SELEZIONA NUOVO PADRE};
				\node (trovato) at (0,-6) [draw, decision,
				] {MAGICO?};
				\node (end) at (0,-8) [draw, terminal,
				minimum width=\smbwd] {FINE};

				\draw[->] (start) -- (generapop);
				\draw[->] (generapop) -- (selgen);
				\draw[->] (selgen) -- (crossover);
				\draw[->] (crossover) -- (muta);
				\draw[->] (muta) -- (trovato);
				\draw[->] (trovato) --node[left]{SI} (end);
				\draw[->] (trovato) --node[above]{NO} (-2.75, -6)
				-- (-2.75, -1.5) -- (0, -1.5);
				\end{tikzpicture}
			}
		\end{figure}

	\end{columns}
\end{frame}

\begin{frame}{algoritmo di Xie e Kang}
%		\begin{block}{}
			\parbox{0.98\columnwidth}{
			L'algoritmo di Xie e Kang\footnotemark[1] è un algoritmo evolutivo per la costruzione di quadrati magici normali con:
			\begin{itemize}
				\item Mutazioni dinamiche e adattive
				\item Rettificazioni locali
				\item Congettura della costruzione a due fasi
			\end{itemize}
		}
%		\end{block}

		\begin{alertblock}{Congettura della costruzione a due fasi}
		\parbox{0.98\columnwidth}{
Un quadrato semimagico è sempre completabile ad un quadrato magico utilizzando un numero finito di permutazioni di righe e di colonne oppure di rettificazioni locali.
		}
		\end{alertblock}
		
		\footnotetext[1]{Xie, T. e Kang, L. (2003), \emph{An Evolutionary Algorithm for Magic Squares}, The 2003 Congress on Evolutionary Computation, 2003.} 
\end{frame}

\begin{frame}{algoritmo di Xie e Kang -- codifica dell'individuo}
    \begin{block}{Individuo}
    	\parbox{0.98\columnwidth}{
	    	Un individuo è una coppia di matrici $ (M,\Sigma) $, la prima è il quadrato da rendere magico, la seconda contiene informazioni necessarie per le mutazioni.
    	}
    \end{block}	
     \begin{block}{Fitness}
     	\parbox{0.98\columnwidth}{
			\[ f(M) = 
			\begin{cases}
			\sum_{i = 1}^{N}\left(\text{row}(i) + \text{col}(i) \right) & \text{semimagico} \\
			- \left(\text{dg1} + \text{dg2} \right) & \text{altrimenti} 
			\end{cases}
			\]
		Dove col$ (i) $ e row$ (j) $ sono rispettivamente la somma degli elementi sulla $ i- $esima colonna e $ j- $esima riga di $ M $, e dg1 e dg2 sono la somma degli elementi sulla diagonale e sull'antidiagonale di $ M $.
%		La fitness è negativa per quadrati semimagici, ciò permette di trovare anche in una popolazione mista il quadrato più vicino ad essere magico.
     	}
     \end{block}	
\end{frame}

\begin{frame}{algoritmo di Xie e Kang -- mutazioni}

%  \begin{columns}[onlytextwidth]

%    \column{0.5\textwidth}

    \begin{block}{Mutazioni}
      \parbox{0.98\columnwidth}{
        \begin{itemize}
        \item Dinamiche: la probabilità di mutazione non è fissa, ma dipende dal numero di linee non magiche
        \item Adattive: le mutazioni dipendono da quanto il quadrato non è magico
        \end{itemize}
      }
    \end{block}
    
    \begin{block}{Mutazioni}
    	\parbox{0.98\columnwidth}{
    		\begin{itemize}
    			\item \alert{Mutazioni puntuali} per quadrati generici
    			\item \alert{Mutazioni lineari} per quadrati che hanno solo le diagonali non magiche (\alert{quadrati semimagici})
    		\end{itemize}
    	}
    \end{block}   
     
%    \column{0.5\textwidth}

%  \end{columns}
\end{frame}

\begin{frame}{algoritmo di Xie e Kang -- mutazioni puntuali I}
	
%	\begin{columns}[onlytextwidth]
		
%		\column{0.45\textwidth}
		
    \begin{block}{Insiemi di mutazione}
    	\parbox{0.98\columnwidth}{
    		\begin{itemize}
    			\item $ S_1 $ numeri la cui riga e colonna non è magica
    			\item $ S_2 $ numeri in righe o colonne non magiche
    		\end{itemize}
    	}
    \end{block} 
    
%        \column{0.5\textwidth}
        
    \begin{block}{Mutazioni}
    	\parbox{0.98\columnwidth}{
    		Siano $ n_{col} $ e $ n_{row} $ il numero di colonne e di righe non magiche. Le mutazione sono scambi di numeri tra:
    		\begin{itemize}
    			\item $ S_1 $ e $ S_2 $ con probabilità $ 1/(n_{row}n_{col}) $.
    			\item $ S_2 $ e $ S_2 $ con probabilità $ P_M $.
    			\item $ S_2 $ e $ M $ con probabilità $ P_M $.
    		\end{itemize}
    		dove 
    		\vspace*{-15pt}
    		\[ 
    		P_M(x) = 
    		\begin{cases}
    		1\slash n_{row} & \text{se} \quad x \text{ è in una riga non magica} \\
    		1\slash n_{col} & \text{se} \quad x \text{ è in una colonna non magica} \\
    		1\slash \left(n_{row} n_{col} \right) & \text{se} \quad x \text{ è in entrambe} \\
    		\end{cases}
    		\]
    	}
    \end{block} 
    
%\end{columns}

\end{frame}

\begin{frame}{algoritmo di Xie e Kang -- mutazioni puntuali II}

    \begin{block}{Esempio: $ S_1 $ in $ S_2 $ }
    	\parbox{0.98\columnwidth}{
    		Siano $ m_{ij} \in M $ e $ \sigma_{ij} \in \Sigma $
    		\begin{enumerate}
    			\item Calcolo $ new =  m_{ij}   + \text{randint}(-\sigma_{ij}, \sigma_{ij})  $
    			\item Aggiusto se è invalido:
    			\[
    			\begin{cases}
    			new = \text{randint}(1,N) & \text{se} \quad new < 1  \\
    			new = N^2 - \text{randint}(0,N) & \text{se} \quad new > N^2  
    			\end{cases}
    			\]
    			\item Cerco l'elemento in $ S_2 $ che più si avvicina a $ new $ cioè $ t \in  S_2  $ tale che soddisfi $ \min_{t \in S_2} \lvert new - t \rvert $.
    			\item Scambio $ t $ e $ new $ in $ M $.	
    		\end{enumerate}
    	}
    \end{block} 
\end{frame}

\begin{frame}{algoritmo di Xie e Kang -- mutazioni puntuali III}
	
	\begin{block}{Esempio: $ S_1 $ in $ S_2 $ }
		\parbox{0.98\columnwidth}{
			\begin{enumerate}
				\setcounter{enumi}{4}
				\item Calcolo $ z =  \sigma_{ij}   + \text{randint}(-1,1)  $
				\item Aggiusto se è invalido:
\[
z = \text{randint}(1,\sigma_t) \quad \text{se} \quad z < 1 \quad \text{o} \quad  z > \sigma_t 
\]
				con:
				\[ \sigma_t = 
				\begin{cases}
				\lvert f(M) \rvert \slash \left(n_{row} + n_{col}\right) & \text{se} \quad n_{row}n_{col} \not = 0  \\
				\lvert f(M) \rvert \slash n_{diag} & \text{se} \quad n_{row}n_{col} = 0 
				\end{cases}
				\]
				$ \sigma_t $ piccola: quadrato quasi magico
				\item Sostituisco a $ \sigma_{ij} $ in $ \Sigma $ il valore $ z $.
			\end{enumerate}
		}
	\end{block} 
	
\end{frame}

\begin{frame}{algoritmo di Xie e Kang -- mutazioni lineari}
	\begin{block}{Mutazioni lineari}
		\parbox{0.97\columnwidth}{
			 Sono permutazioni casuali di una linea di un quadrato semimagico.
			 \vspace*{-15pt}
			 \begin{enumerate}
				\item Seleziono un numero $ q $ intero da 1 a $ N $.
				\item Per $ q $ volte estraggo una linea.
				\item La sostituisco con una permutazione casuale dei suoi elementi.
			\end{enumerate}
			La linea rimane magica.
		}
	\end{block}
\end{frame}

\begin{frame}{algoritmo di Xie e Kang -- rettificazioni locali}
	\begin{block}{Rettificazioni locali}
		\parbox{0.98\columnwidth}{
			L'algoritmo potrebbe rimanere in una fase di stallo, per questo conviene operare con approcci sistematici:
			\begin{itemize}
				\item \alert{Rettificazioni lineari}: cercano di aumentare il numero di linee magiche.
				\item \alert{Rettificazioni diagonali}: cercano di rendere le diagonali di un quadrato semimagico magiche.
			\end{itemize}
			Le rettificazioni sono ottenute analizzando tutto il quadrato in cerca di tutte le coppie o i quartetti tali che una loro permutazione migliori il quadrato. 
		}
	\end{block}
\end{frame}

\begin{frame}{algoritmo di Xie e Kang -- rettificazioni locali lineari}
			
	  \begin{columns}[onlytextwidth]

	    \column{0.5\textwidth}		
		
		\begin{block}{Esempio di rettificazione locale lineare}
		\parbox{0.98\columnwidth}{
		Due numeri $ m_{ks} $ e $ m_{ls} $ sono scambiati alla riga $ k $ e $ l $ e alla colonna $ s $ se sono soddisfatte: 
		\begin{itemize}
			\item $ \text{row}(k) - \mathit{m.v.} = m_{ks} - m_{ls} $
			\item $ \mathit{m.v.} - \text{row}(l) = m_{ks} - m_{ls} $
		\end{itemize}
		con $ m.v. $ costante magica.
		}
		\end{block}
		Sono state implementate altre tre condizioni simili, che coinvolgono due o quattro numeri.
		
		\column{0.5\textwidth}
		
		\begin{figure}[htbp]
			\centering
			\subfloat[Prima]{
				\begin{tikzpicture}[scale = 0.75]
				\node at (-0.5, 0) {};				
				\draw (0,0) grid (3,3);
				\node at (1.5, 1.5) {\large 3};
				\node at (0.5, 0.5) {\large 2};
				\node at (1.5, 0.5) {\large 7};
				\node at (0.5, 2.5) {\large 1};
				\node at (0.5, 1.5) {\large 4};
				\node at (2.5, 2.5) {\large 6};
				\node at (1.5, 2.5) {\large 5};
				\node at (2.5, 0.5) {\large 9};
				\node at (2.5, 1.5) {\large 8};
				\draw[->, thick] (3.25, 2.5) -- (3.5, 2.5) node[right] {\large 12};
				\draw[->, thick] (3.25, 1.5) -- (3.5, 1.5) node[right] {\large 15};
				\draw[->, thick] (3.25, 0.5) -- (3.5, 0.5) node[right] {\large 18};
				\end{tikzpicture}		
			}
			
			\subfloat[Dopo]{
				
				\begin{tikzpicture}[scale = 0.75]
				\node at (-0.5, 0) {};				
				
				\draw (0,0) grid (3,3);
				
				\draw[fill = gray, opacity = 0.2]
				(2,2) rectangle (3,3);	
				
				\draw[fill = gray, opacity = 0.2]
				(2,1) rectangle (3,0);	
				
				\node at (1.5, 1.5) {\large 3};
				\node at (0.5, 0.5) {\large 2};
				\node at (1.5, 0.5) {\large 7};
				\node at (0.5, 2.5) {\large 1};
				\node at (0.5, 1.5) {\large 4};
				\node at (2.5, 2.5) {\large \textbf{9}};
				\node at (1.5, 2.5) {\large 5};
				\node at (2.5, 0.5) {\large \textbf{6}};
				\node at (2.5, 1.5) {\large 8};
				\draw[->, thick] (3.25, 2.5) -- (3.5, 2.5) node[right] {\large 15};
				\draw[->, thick] (3.25, 1.5) -- (3.5, 1.5) node[right] {\large 15};
				\draw[->, thick] (3.25, 0.5) -- (3.5, 0.5) node[right] {\large 15};
				\end{tikzpicture}			
			}
			
		\end{figure}
		
	\end{columns}
\end{frame}

\begin{frame}{algoritmo di Xie e Kang -- rettificazioni locali diagonale}
	
	\begin{columns}[onlytextwidth]
		
		\column{0.55\textwidth}	
		Rettificazioni locali diagonali:
		\begin{itemize}
			\item Puntuali: se scambiano numeri
			\item Lineari: se scambiano linee
		\end{itemize}			
		\begin{block}{Esempio di rettificazione locale diagonale puntuale}
			\parbox{0.96\columnwidth}{
			Se sono soddisfatte le condizioni:
			\begin{itemize}
				\item $ a_{ii} + a_{ij} = a_{ji} + a_{jj} $
				\item $ \left(a_{ii} +  a_{jj} \right) - \left(a_{ij} +  a_{ji} \right) = \text{dg1} - \mathit{m.v.}$
			\end{itemize}
			allora $ a_{ii} $ è scambiato con $ a_{ji} $ e $ a_{ji} $ con $ a_{jj} $. 
			}
		\end{block}
		
		\column{0.5\textwidth}
		
		\begin{figure}[htbp]
			\centering
			\subfloat[Prima]{
				\begin{tikzpicture}[scale = 0.75]
				\node at (-1, 0) {};				
				\draw (0,0) grid (3,3);
				\node at (1.5, 1.5) {\large 2};
				\node at (0.5, 0.5) {\large 8};
				\node at (1.5, 0.5) {\large 5};
				\node at (0.5, 2.5) {\large 4};
				\node at (0.5, 1.5) {\large 7};
				\node at (2.5, 2.5) {\large 9};
				\node at (1.5, 2.5) {\large 1};
				\node at (2.5, 0.5) {\large 3};
				\node at (2.5, 1.5) {\large 6};
				
				\draw[->, thick] (3.25, -0.25) --(3.5, -0.5);
				\node at (3.75, -0.75) {\large 9};   
				\end{tikzpicture}		
			}
			
			\subfloat[Dopo]{
				
				\begin{tikzpicture}[scale = 0.75]
				\node at (-1, 0) {};				
				
				\draw (0,0) grid (3,3);
				
				\draw[fill = gray, opacity = 0.2]
				(1,0) rectangle (3,2);	
				
				\node at (1.5, 1.5) {\large \textbf{5}};
				\node at (0.5, 0.5) {\large 8};
				\node at (1.5, 0.5) {\large \textbf{2}};
				\node at (0.5, 2.5) {\large 4};
				\node at (0.5, 1.5) {\large 7};
				\node at (2.5, 2.5) {\large 9};
				\node at (1.5, 2.5) {\large 1};
				\node at (2.5, 0.5) {\large \textbf{6}};
				\node at (2.5, 1.5) {\large \textbf{3}};
				
				\draw[->, thick] (3.25, -0.25) --(3.5, -0.5);
				\node at (3.75, -0.75) {\large 15};   
				\end{tikzpicture}			
			}
			
		\end{figure}
		
	\end{columns}
\end{frame}

\begin{frame}{algoritmo di Xie e Kang -- metodi di selezione}
	\begin{block}{Metodi di selezione}
		\parbox{0.97\columnwidth}{
			Il nuovo genitore è 
			\begin{itemize}
				\item$ (\mu, \lambda)-ES $: il migliore figlio della precedente. Permette maggiore variabilità.
				\item $ (\mu + \lambda)-ES $: il migliore tra il genitore e i figli della precedente. Permette di conservare i risultati ottenuti.
			\end{itemize}
			I metodi applicati dipendono dalle caratteristiche del quadrato migliore.
		}
	\end{block}
\end{frame}

\begin{frame}[fragile]{algoritmo di Xie e Kang -- alcuni punti delicati o interessanti I}
	Funzione di fitness:
	
\begin{lstlisting}[language=Mathematica,basicstyle=\ttfamily,keywordstyle=\color{blue}]	
<@\textcolor{gray}{(*La fitness e' negativa quando il quadrato e' }@>
<@\textcolor{gray}{semimagico, questo mi permette di renderli preferiti}@>
<@\textcolor{gray}{ai quadrati generici*)}@>
If[incorrectLines[ind] === 0,
   Return[-Total[diagonalsDeviation[ind]]],
   Return[Total[linesDeviation[ind]]];
];
\end{lstlisting}
\vspace*{10pt}
Estensione dei metodi su tutta la popolazione:
\vspace*{-10pt}
\begin{lstlisting}[language=Mathematica,basicstyle=\ttfamily,keywordstyle=\color{blue}]	
fitnessPop[pop_List] := Return[fitness /@ pop];
\end{lstlisting}
\end{frame}

\begin{frame}[fragile]{algoritmo di Xie e Kang -- alcuni punti delicati o interessanti II}
	\texttt{fittestChild}: selezionare l'individuo migliore in \texttt{pop}.

\begin{lstlisting}[language=Mathematica,basicstyle=\ttfamily,keywordstyle=\color{blue}]	
fp = fitnessPop[pop];
<@\textcolor{gray}{(*Controllo se c'e' l'individuo perfetto*)}@>
If[Length[Position[fp, 0]] =!= 0,
   Return[pop[[Position[fp, 0][[1,1]]]]];
];
<@\textcolor{gray}{(*Io voglio l'individuo piu' vicino a zero,}@>
<@\textcolor{gray}{ma voglio anche privilegiare chi ha fitness negativa*)}@>
If[Min[fp] < 0,
   fp = (#)^(-1) & /@ fp
];
<@\textcolor{gray}{(*Posizione del migliore*)}@>
Random[Integer, {1, Length[Position[fp, Min[fp]]]}]
\end{lstlisting}
\end{frame}

\begin{frame}[fragile]{algoritmo di Xie e Kang -- alcuni punti delicati o interessanti III}
Algoritmo completo:
	
\begin{lstlisting}[language=Mathematica,basicstyle=\ttfamily,keywordstyle=\color{blue}]	
[...] <@\textcolor{gray}{(*Vari controlli*)}@>
offspring = <@\textcolor{blue}{ParallelMap}@>[mutate, offspring];
[...] <@\textcolor{gray}{(*Altri controlli*)}@>
[...] <@\textcolor{gray}{(*Salvo risultati intermedi*)}@>
\end{lstlisting}
L'algoritmo completo esegue operazioni che coinvolgono la popolazione intera parallelizzate su una macchina (lara) dotata di quattro core. 

\end{frame}

\begin{frame}{algoritmo di Xie e Kang -- risultati}

\begin{table}[htbp]
	\centering
	\caption*{Risultati con $ N $ ordine del quadrato, $ n_{tent} $ numero di tentativi di esecuzione, $ n_{ok} $ numero di successi e $ \tau $ tempo medio di esecuzione.}
	\vspace*{-10pt}
	\alt<2->{\newcolumntype{C}{>{\columncolor{green!25}}c}}{\newcolumntype{C}{c}}
	\subfloat[Popolazione di 25 figli.]{
		\begin{tabular}{c|CCc}	\toprule
			$ N $ & $ n_{tent} $ & $ n_{ok} $ &        $ \tau $        \\ \midrule
			3   &      10      &     10     & $ \SI{0.12}{\second} $ \\
			10   &      10      &     10     &  $ \SI{55}{\second} $  \\
			15   &      10      &     10     & $ \SI{5.75}{\minute} $ \\
			20   &      10      &     10     & $ \SI{31.2}{\minute} $ \\
			25   &      10      &     10     &  $ \SI{1.73}{\hour} $  \\
			30   &      10      &     10     &  $ \SI{4.23}{\hour} $  \\
			35   &      10      &     10     & $ \SI{12.38}{\hour} $  \\
			40   &      10      &     10     & $ \SI{25.79}{\hour} $  \\ \bottomrule
		\end{tabular} 
	}   \qquad
	\subfloat[Popolazione di 10 figli.]{	
		\begin{tabular}{c|CCc}	\toprule
			$ N $ & $ n_{tent} $ & $ n_{ok} $ &        $ \tau $        \\ \midrule
			3   &      10      &     10     & $ \SI{0.22}{\second} $ \\
			10   &      10      &     10     &  $ \SI{69}{\second} $  \\
			15   &      10      &     10     & $ \SI{4.37}{\minute} $ \\
			20   &      10      &     10     & $ \SI{26.3}{\minute} $ \\
			25   &      10      &     10     &  $ \SI{1.51}{\hour} $  \\
			30   &      10      &     10     &  $ \SI{3.97}{\hour} $  \\
			35   &      10      &     10     &  $ \SI{8.63}{\hour} $  \\
			40   &      10      &     10     & $ \SI{15.73}{\hour} $  \\ \bottomrule
		\end{tabular} 
	}
\end{table}
\centering
	\only<2->{L'algoritmo non ha \alert{mai} fallito.}
\end{frame}

\begin{frame}{algoritmo di Xie e Kang -- fitness}
	\centering
	\includegraphics*[width=0.9\columnwidth]{fitvsgen}
\end{frame}

\begin{frame}{algoritmo di Xie e Kang -- linee non magiche}
	\centering
	\includegraphics*[width=0.9\columnwidth]{incvsgen}
\end{frame}

\begin{frame}[fragile]{algoritmo di Xie e Kang -- tempi di esecuzione}

\begin{figure}[htbp]
	\centering
	\begin{tikzpicture}
	\begin{axis}[
	axis lines=left, enlarge x limits=true,
	mlineplot,
	%	xtick={10,15,20,25,30},
	xtick = data,
	scaled y ticks=base 10:-3,
	ymin=0,
	xmin=7.5,
	title = Tempo di esecuzione vs ordine,  
	xlabel = Ordine,
	ylabel = Tempo di esecuzione,
	grid=major,
	grid style={dashed,gray!30},	
	y unit=\si{\second},
	width=.75\columnwidth,
	title style={yshift=1.5ex},
	]
	\addplot [only marks, error bars/.cd,
	y dir=both, y explicit]
	table[x=x, y=y, y error=error]{
		x  y	    error
		10 55.2336 10.385984
		15 345.519 131.58943
		20 1873.96 629.16364
		25 6229.42 1673.8868
		30 15222.0 3930.1511
		35 45535.3 13768.331
		40 76630.5 22423.122
	};
	\addplot [only marks, mark=o, error bars/.cd,
	y dir=both, y explicit]
	table[x=x, y=y, y error=error]{
		x  y	    error
		10 69.3912 16.0766
		15 262.52  38.9824
		20 1578.77 542.591
		25 5424.71 1484.1
		30 14297.517 3034.074
		35 31085.711  6122.3081
		40 56637.936  11245.582
	};
	\addplot [blue, thick, smooth] table{
		10 55.2336
		20 1100
		30 4442
		40 16060
	};
	\addplot[red,domain=1:40,samples=100, thick] {0.0002202440975*x^(5.334925044)};
	\addplot[green,domain=1:40,samples=100, thick]
	{0.000463778412*x^(5.045827779)};
	\node at (axis cs:32, 1500) [anchor=west, rotate = 20] {\small Xie Kang};
	\node at (axis cs:36, 55000) [anchor=west, rotate = 65] {\small pop = 25};
	\node at (axis cs:36, 25000) [anchor=west, rotate = 53] {\small pop = 10};
	\end{axis}
	\end{tikzpicture}
\end{figure}
\end{frame}

\begin{frame}[fragile]{algoritmo di Xie e Kang -- distribuzione dei tempi di esecuzione I}

%\only<1>{	
	\begin{table}[htbp]
		\centering
		\caption*{Distribuzione dei tempi di esecuzione}
		\begin{tabular}{cc}
			\toprule	
			Routine          & Tempo speso ($ \si{\percent} $) \\ \midrule
			\texttt{selectFittest}   &          $ \sim 0.01 $          \\
			\texttt{mutate}      &          $ \sim 0.7 $           \\
			\texttt{rectifyDiagonals} &          $ \sim 7.7 $           \\
			\texttt{rectifyLines}   &           $ \sim 91 $           \\
			\verb|/   \| &   \verb|/ \| \\
			\begin{tabular}{c|c}
				\texttt{onePair}  &   \texttt{twoPairs} 
			\end{tabular} 	   		  & \begin{tabular}{c|c}
			$ \sim 13 $   &   $ \sim 87 $ 
		\end{tabular} 		\\
		\bottomrule
	\end{tabular} 
\end{table}
%}
%\only<2->{
%	
%	\begin{tikzpicture}
%	% explode all
%	\pie[ pos ={0,0}, explode =0.1]{10/ A , 20/ B , 30/ C, 40/ D }
%	\end{tikzpicture}
%
%}
\end{frame}

\begin{frame}[fragile]{algoritmo di Xie e Kang -- distribuzione dei tempi di esecuzione II}
	
Perché? (Rettificazioni a due coppie)
\begin{lstlisting}[language=Mathematica,basicstyle=\ttfamily,keywordstyle=\color{blue}]
<@\textcolor{gray}{(*i1,j1 indici del primo elemento*)}@>
For[i1 = 1, i1 <= N, i1++,  
    For[j1 = 1, j1 <= N, j1++, 
        <@\textcolor{gray}{(*Scorro il resto del quadrato dopo i1,j1*)}@>
        For[i2 = i1 + 1, i2 <= N, i2++, 
            For[j2 = 1, j2 <= N, j2++, 
               [...]  <@\textcolor{gray}{(*Vari controlli*)}@>
\end{lstlisting}
Il numero di operazioni cresce come $ N^4 $. \\
%Le rettificazioni a una coppia hanno un ciclo in meno perché si controllano solo gli elementi sulla stessa linea. \\
E' necessaria questa implementazione perché bisogna operare direttamente con gli indici.
\end{frame}

%\section{conclusioni e sviluppi futuri}

\begin{frame}{conclusioni e sviluppi futuri -- conclusioni}
%	\begin{block}{Conclusioni}
		\begin{itemize}
%			\parbox{0.95\columnwidth}{
			\item \alert{E' possibile} realizare con successo algoritmi stocastici per la costruzione di quadrati magici se non si usano crossover e se si interviene in modo sistematico
			\item Questi algoritmi sono molto più efficienti della ricerca a tappeto
			\item Si è mostrato che l'approccio di Xie e Kang \alert{funziona}
			\item \alert{Non} si è ritrovata la legge di scala trovata da Xie e Kang, probabilmente a causa della quasi totale ignoranza riguardo l'implementazione originale
%			\item Mathematica \alert{non} si è rivelato necessario ai fini dell'implementazione 
%			}
		\end{itemize}
%	\end{block}
\end{frame}

\begin{frame}{conclusioni e sviluppi futuri -- generalizzazioni e sviluppi futuri}
	%	\begin{block}{Conclusioni}
	Alcune questioni lasciate aperte:
	\begin{itemize}
		%			\parbox{0.95\columnwidth}{
		\item Ottimizzare implementazione, \texttt{Compile[]}?
		\item Confrontare con realizzazione in linguaggio compilato
		\item Parallelizzare i metodi di rettificazione
		\item Indagare dipendenza del tempo di esecuzione dalla dimensione della popolazione
		%			}
	\end{itemize}
	Possibili estensioni:
	E' possibile generalizzare l'algoritmo fintato che si generalizzano i metodi di rettificazione, quindi per tutti quei casi in cui $ m.v. $ è fissato.
	Ad esempio:
	\begin{itemize}
		\item Quadrati magici vincolati
		\item Quadrati magici non normali con costante magica fissata
	\end{itemize}
	%	\end{block}
\end{frame}

\begin{frame}{conclusioni e sviluppi futuri -- il ruolo di Mathematica}
	%	\begin{block}{Conclusioni}
	Mathematica \alert{non si è rivelato necessario} nell'implementazione di questi algoritmi perché la quasi totalità della manipolazione è numerica e non simbolica. \\
	Tuttavia Mathematica ha reso l'implementazione più diretta perché:
	\begin{itemize}
		%			\parbox{0.95\columnwidth}{
		\item Le funzioni vettorializzate, come \texttt{Map[]}, \texttt{Apply[]}, \texttt{Table[]} permettono l'implementazione molto elegante dei metodi che agiscono sulle popolazioni intere
		\item \texttt{Replace[]} permette di fare gli scambi in modo conciso è chiaro
		%			}
	\end{itemize}
	Il costo di questa semplificazione è probabilmente la riduzione nell'efficienza.
	
	%	\end{block}
\end{frame}

\begin{frame}[standout]
	GRAZIE PER L'ATTENZIONE!
\end{frame}

\end{document}
